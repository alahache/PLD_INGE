
\documentclass{mise_en_page}

\projet{Projet Ingéniérie}
\equipe{H4314}
\responsable{Tristan Delizy}
\redacteurs{Pierre LULÉ, Léo LEFEBVRE, Sebastien LAURENT, Jérémy TULOUP} % Optionnel
\titre{Étude de faisabilité}
\version{1.1}
\objet{Présenter de manière succincte l’existant, ses points forts et
ses faiblesses. Il doit présenter une étude de faisabilité par rapport
aux technologies émergentes et fiables, et donner quelques pistes
d’évolution.}
\etat{non validé} %draft, non relu, non validé, validé

\begin{document}

\maketitle

\begin{historique}
%    \histo{0.1}{xx/xx/2012}{Draft initial}
    \histo{1.0}{16/01/2012}{Première version}
    \histo{1.1}{23/01/2012}{Corrections mineures}
\end{historique}

\newpage

\tableofcontents

\section{Introduction}
Le COPEVUE (Comité pour la protection de l’EnVironnement de l’UE)
souhaite étudier un système de monitoring à distance de sites isolés.
Son rôle est de permettre le ravitaillement de stations-réservoirs de
sites isolés.

\section{Analyse de l’existant}
\subsection{Analyse du métier}
Orienté sur la gestion des différents sites isolés, la part métier de ce
que l’on peut considérer comme existant se limite simplement à la
connaissance des conditions d’accès aux sites.

\subsection{Analyse des savoir-faire et des processus}
\subsubsection{Surveillance des niveaux}
Actuellement, la surveillance des différents lieux n’est basée sur aucun
système automatisé. Le relevé, c’est-à-dire les mesures du niveau
d’essence, d’eau ou de déchets selon le type de réservoir, sont faits
manuellement par le propriétaire du lieu de travail. Ensuite, suivant
le niveau constaté il avertit directement la société en charge de
s’occuper du réservoir afin qu’elle vienne le remplir ou bien le vider.

Il y a plusieurs inconvénients vis-à-vis de cette méthode. Tout d’abord,
les relevés doivent être faits périodiquement, pour chaque réservoir,
et mesurés manuellement. Outre le fait que cette tâche peut vite
devenir lassante, elle est sujet à risques notamment à cause de la
qualité de la mesure. De plus, les sociétés intervenantes sont
seulement prévenues dans le cas du dépassement d’un seuil critique.
Cela a de fortes chances d'entraîner des problèmes de
logistique et d’optimisation des déplacements. 


\subsubsection{Logistique et déplacements}
Les différentes sociétés en charge de la maintenance des réservoirs sont
donc prévenues à chaque fois que le propriétaire du lieu fait une
demande de remplissage ou de vidage. Suite à cette demande, la société
prépare et envoie un camion avec tous les équipements nécessaire pour
réaliser cette demande au lieu dit.  Il n’y a donc aucune anticipation
par rapport au signal donné par le propriétaire du lieu.

Dans le cas où la société est à la charge de plusieurs sites, on
remarque tout de suite que cette méthode n’est pas optimale. Les coûts
de transports croient très vite, influençant directement la qualité de
prestation, et créant ainsi de nouveaux risques environnementaux.

\subsection{Analyse du matériel utilisé}
Le relevé des niveaux se fait selon la façon du propriétaire du lieu
par ses propres moyens, nous n’avons aucune information sur la
précision, sur la qualité et sur la rigueur des mesures.

Pour la demande de maintenance, le propriétaire du lieu utilise son
téléphone pour appeler directement la société en charge (appel ou SMS)
de l’intervention. Celle-ci apporte directement la marchandise par
camion.

\section{Étude de faisabilité}
\subsection{Synthèse sur le système embarqué}
Le système embarqué (ie : machine physique déployée sur site) doit être
capable de gérer un réseau de capteurs local (filaire ou non) ainsi
qu’une connexion distante lui permettant à la fois d’émettre des
données, mais aussi d’en recevoir (pour procéder à une mise à jour par
exemple). Le tout doit être autonome énergétiquement ce qui impose donc
de dimensionner le système au plus près afin de limiter toute
surconsommation inutile.

La mise en place d’une solution “sur mesure”, c’est à dire de fabriquer
sa propre carte mère électronique, serait trop coûteuse à la fois en
terme de déploiement (étude préalable et fabrication d’une population
trop faible pour amortir suffisamment les coûts) mais aussi en terme de
développement logiciel à fournir par la suite.

Il est donc essentiel d'utiliser des solutions
existantes favorisant, sur le plan matériel, la modularité et la
robustesse. En effet, la configuration risque de changer selon la
nature du site, il est donc important de pouvoir choisir quels
composants seront à implémenter. De même,
l'environnement est aussi une variable à prendre en
compte, et les appareils utilisés doivent être parfois aptes à
fonctionner dans des conditions de température et
d'humidité extrêmes.

Dans le cadre de cette étude de faisabilité, nous avons examiné les
produits Arduino, constructeur de circuits imprimés. Ce dernier propose
à un tarif abordable une solution répondant à ces critères. Nous
pouvons donc conclure qu'il existe actuellement sur le
marchés de tels systèmes. Une étude de marchés plus approfondie sera
cependant nécessaire pour effectuer un choix judicieux entre les divers
fabriquants.

\subsection{Synthèse sur la gestion de l’énergie}
La problématique est complexe lorsqu’il s’agit de gérer l’accès à
l’énergie dans des lieux isolés. Retenons qu’il est nécessaire
d’alimenter à la fois chaque capteur mais aussi le système embarqué. La
solution retenue dépendra donc, en principe, de l’implantation
effective du site.




\subsubsection{Cas 1 (raccordement filaire du réseau de capteurs) :}

Pour ce type de sites, une unique batterie centralisée sera installée au
niveau du système embarqué, reliée par la suite à chaque capteur au
moyen de fils. Le raccordement à un coût unique de mise en oeuvre qui
peut être amorti sur du moyen terme. Cette batterie pourra alors subir
deux types de maintenances au choix : 

\begin{enumerate}
\item Remplacement de celle-ci par une batterie chargée par un opérateur
de maintenance (opération régulière et facile mais coûteuse à la
longue).
\item Rechargement de celle-ci par un système de production adéquat ou
un raccordement au réseau le cas échéant. Ce genre d’alimentation
autonome devra être étudiée au cas par cas pour étudier sa rentabilité
vis à vis du premier type de maintenance (à savoir le déplacement d’un
opérateur).
\end{enumerate}



\subsubsection{Cas 2 (raccordement sans fil du réseau de capteurs) :}

Pour les sites concernés, chaque capteur possède sa propre alimentation
par batterie (piles). L’opération de changement et/ou rechargement de
celles-ci à un coût important à long terme justifié par l’intervention
régulière d’un opérateur (déplacement difficile). Il faut de plus noter
que cette opération de maintenance n’a pas de lien systématique avec
celle de livraison effectuée par camions et doit donc en être
dissociée.

\subsection{Synthèse sur les capteurs}

Les capteurs que nous choisirons doivent pouvoir résister à des
conditions extrêmes comme c’est le cas au Nord de la Norvège avec des
températures pouvant atteindre les -80 °C !

Nous devons donc sélectionner nos capteurs suivant plusieurs critères
comme leur résistance aux hautes températures, à de fortes pression.
Ils doivent également être résistant au substances mesurées (rouille ou
corrosion pour de l’eau ou de l’essence).




Pour notre système, il nous faudra trouver des capteurs et moyens
permettant de détecter plusieurs paramètres

\begin{itemize}
\item mesurer les flux, 
\item mesurer la températures et la pression des cuves ou du site pour
assurer une bonne sécurité,
\item détecter les incendie,
\item détecter la présence d’intrus.
\end{itemize}



Après une recherche sur différents types de capteur et de moyens
disponibles pour assurer ces fonctions nous avons dressé les résultats
suivant :

\emph{Remarque : dans les tableau qui suivent :}
\begin{itemize}
\item  “ ++ “ signifie que le capteur répond au critère correspondant.
\item  “ 0 “ signifie que le capteur répond moyennement au critère
correspondant.
\item  “ -- “ signifie que le capteur ne répond pas au
critère correspondant.
\end{itemize}

\begin{flushleft}
\tablehead{}
\begin{supertabular}{|m{2.71cm}|m{1.308cm}|m{2.155cm}|m{1.8109999cm}|m{1.8369999cm}|m{2.816cm}|m{2.472cm}|}
\hline
Critère &
Radar &
Radar à impulsion guidées &
Ultrason &
Capacitif &
Hydrostatique &
Radiométrie\\\hline
Matériaux solide &
++ &
++ &
++ &
++ &
{}-{}- &
{}-{}-\\\hline
Matériaux liquide &
++ &
++ &
++ &
++ &
++ &
++\\\hline
Précision des mesures &
++ &
++ &
0 &
0 &
0 &
++\\\hline
Prix &
++ &
++ &
++ &
++ &
++ &
++\\\hline
Maintenance à réaliser &
++ &
++ &
0 &
++ &
++ &
++\\\hline
Résistances aux températures et pressions extrèmes &
++ &
++ &
++ &
++ &
++ &
++\\\hline
\end{supertabular}
\end{flushleft}



Table comparative des capteurs de niveau




\begin{flushleft}
\tablehead{}
\begin{supertabular}{|m{2.843cm}|m{2.261cm}|m{2.552cm}|m{2.552cm}|m{2.552cm}|m{2.552cm}|}
\hline
~
 &
Capteurs de proximité capacitifs &
Capteurs de proximité inductifs &
Capteurs infra-rouges &
Capteurs de détection par pression &
Analyse d’image par caméra standard\\\hline
Détection indépendante 

 de la cible &
0 &
0 &
++ &
++ &
++\\\hline
Grande distance de détection  &
0 &
0 &
++ &
0 &
++\\\hline
Effet de la luminosité &
++ &
++ &
0 &
++ &
0\\\hline
Effet de la température de la cible &
++ &
++ &
0 &
++ &
++\\\hline
Prix &
++ &
++ &
++ &
0 &
0\\\hline
\end{supertabular}
\end{flushleft}



Table comparative des capteurs pour la détection




\begin{flushleft}
\tablehead{}
\begin{supertabular}{|m{5.303cm}|m{5.303cm}|m{5.303cm}|}
\hline
~
 &
Thermomètres infrarouges &
Thermomètres à sonde (platine)\\\hline
Prix &
0 &
++\\\hline
Connectique réseau &
++ &
++\\\hline
Gamme de températures extrêmes &
0 &
++\\\hline
Faible maintenance &
0 &
++\\\hline
\end{supertabular}
\end{flushleft}



Table comparative des capteurs pour la température




\begin{flushleft}
\tablehead{}
\begin{supertabular}{|m{5.303cm}|m{5.303cm}|m{5.303cm}|}
\hline
~
 &
Baromètre numérique &
Baromètre analogique\\\hline
Prix &
++ &
++\\\hline
Résistances aux conditions extrêmes &
++ &
0\\\hline
Connectique pour analyse des données &
++ &
0\\\hline
Faible maintenance &
++ &
0\\\hline
Large gamme de pression mesurable &
++ &
0\\\hline
\end{supertabular}
\end{flushleft}



Table comparative des capteurs pour la pression

Nos choix pour chaque catégorie se tourneront vers :

\begin{description}
\item[capteurs de niveau] : radar
\item[capteurs pour la détection des sites normaux] : une combinaison de
capteurs infra-rouges 
\item[capteurs pour la détection des sites à risque] : une combinaison de
capteurs infra-rouges et caméras
\item[capteurs pour la température] : thermomètre à sonde
\item[capteurs pour la pression] : baromètre numérique
\end{description}

\subsection{Synthèse sur les systèmes de communication}
Le système de communication doit permettre de réaliser la liaison entre
:

\begin{itemize}
\item Les capteurs et le système embarqué de la station.
\item La station et le site distant.
\end{itemize}
Davantage d’informations sur les autres composantes du système de
communication global sont disponibles en annexe.

\subsubsection{Communication entre les capteurs et le système embarqué de la
station}
Le relevé des niveaux des réservoirs s’effectue à l’aide de capteurs (cf
Partie Capteurs). Cette communication entre les capteurs et la station
doit respecter plusieurs critères. Elle doit : 

\begin{itemize}
\item être adaptée à toute condition météorologique
\item permettre de gérer les différents types de capteurs
\item permettre de réaliser une connexion entre deux points situés à une
centaine de mètre l’un de l’autre
\item être fiable
\item être économique
\end{itemize}
Cette communication peut donc être réalisée à l’aide d’une connexion
filaire ou sans fil.

Chacun des deux modes de connexion présente ses avantages et
inconvénients. 

Par exemple, la connexion sans fil permet de s’affranchir des
contraintes de câblage et d'entretien des câbles, et
permet au système embarqué d’être mobile. En revanche, il faut être
capable de couvrir toute la zone, dans le cas où une solution axée sur
la technologie Wi-fi est retenue. En effet les capteurs peuvent être
distant de plusieurs dizaines de mètres les uns des autres.

L’utilisation d’une connexion de type filaire permet en revanche
d’assurer une meilleure fiabilité et un meilleur débit pour les
transmissions d’information. De plus, c’est peut-être une solution
moins coûteuse en terme d’infrastructure et elle est tout à fait
envisageable. La distance entre un réservoir et le système embarqué
n’est que d’une centaine de mètres au maximum. Mais elle ne respecte
pas la contrainte d’entretien lié à l’exigence fonctionnelle des
différents types de météo et climats possibles.

\subsubsection{Communication entre la station et le site distant}
Le système embarqué est chargé de faire parvenir les informations des
différents niveaux de cuves et de réservoirs au serveur central. 

Ces informations peuvent être envoyées simplement lorsque le besoin de
remplir ou de vider une cuve se fait ressentir. Mais cela consistait à
un des points faibles mis en valeur par l’étude de l’existant.

Finalement il est préférable d’avoir un suivi en temps réel sur le
niveau de toutes les cuves de chaque lieu, auquel cas le système devra
à intervalles réguliers envoyer ses données. Toujours est-il qu’il doit
y avoir une liaison entre le système et le serveur distant. 

Cette liaison est également pilotée par plusieurs contraintes :

\begin{itemize}
\item La communication doit être possible depuis n’importe quel endroit
du globe
\item Elle doit supporter différentes conditions météorologique
\item Elle doit être économique sur le plan de l’énergie
\end{itemize}

Cette liaison pourrait être de type filaire. Mais dans notre cas, de
nombreux sites se situent dans des régions pratiquement inaccessibles.
L’installation d’un réseau de communication filaire serait une
contrainte énorme et engendrerait des coût dramatiquement élevés. 

Une solution idéale et tout à fait envisageable est d’utiliser le réseau
GSM (GPRS), qui permet de s’affranchir de toutes les contraintes
précédentes.

Plusieurs systèmes de ce type existent déjà et pourraient convenir à
notre cas d’étude.

\subsection{Synthèse sur systèmes de localisation}
\subsubsection{Pour les agents de maintenance}
La localisation des agents de maintenance la plus adaptée est la
localisation par GPS, technique éprouvée et adaptée à un objet mobile.
Sa précision et sa couverture à l’échelle mondiale permettent à un
véhicule d’un agent de maintenance de rejoindre un site facilement.

\subsubsection{Pour les sites}
La meilleure solution pour les sites isolés est d’effectuer une
géolocalisation (par GPS par exemple) à l’installation du site et de
maintenir une base de donnée des positions des sites. En effet, la
consommation électrique du GPS est trop importante pour être utilisable
en permanence sur un site et est inutile dans le cas général car les
sites sont immobiles.

Il reste envisageable d’offrir exceptionnellement la possibilité d’un
site mobile, localisé par GPS et bénéficiant d’un apport énergétique
particulier.

\subsection{Synthèse sur les OS}
La partie software des système embarqués est extrêmement simple et n’a
que deux besoins fonctionnels :

\begin{itemize}
\item Acquérir des données de capteurs
\item Transmettre ces données
\end{itemize}
À côté de ceci, le système doit être absolument prédictible et fiable.
Ceci nous oriente vers un système d’exploitation temps réel (RTOS).

De ce fait, l’OS doit être le plus simple possible ou être complètement
modulaire (micro-noyau).

On peut d’emblée éliminer les systèmes complexes visant la compatibilité
unix (comme LynxOS ou RTLinux).

Le choix entre les RTOS adaptés (Enea OSE, QNX, VxWorks...) sera
finalement intimement lié au choix du matériel.

On remarque que les systèmes disponibles sur le marché sont rarement
simple mais plus souvent modulaire. Du fait que la modularité augmente
les risques de mauvaise fiabilité (un système modulaire ne peut être
entièrement testé) il ne faut pas exclure non plus le développement
d’un mini-OS parfaitement adapté.

\section{Tendances / Choix}
Globalement, la tendance dégagée est de chercher à allier la simplicité
et la fiabilité. En effet, en situation «extrême», le meilleur choix
est souvent le plus simple, celui qui ne nécessite pas trop de matériel
et d’expertise.

Cette tendance se retrouve à la fois dans :

\begin{description}
\item[le choix du système embarqué] : une simple carte aisément
remplaçable
\item[le choix de l’alimentation énergétique] : les piles et batteries
sont facilement remplaçables et rechargeable, contrairement à un
système du type panneau solaire qui entraînerait des surcoûts
considérables en cas de remplacement.
\item[le choix des capteurs] : un des principaux critère pour les
capteurs est la résistance à une basse température
\item[systèmes de communication] : le choix de système standard permet
une plus grande simplicité de maintenance (une puce wifi défectueuse
peut facilement être remplacée contrairement à un système de
communication spécifique)
\item[système de localisation] : la localisation des sites est la plus
simple et la plus fiable : leurs coordonnées sont enregistrées. La
localisation des agents de maintenance est la plus éprouvée : le GPS.
\item[le choix de l’OS] : la simplicité et la fiabilité sont les deux
critères les plus importants du choix de l’OS
\end{description}

\section{Conclusions}
Un nouveau système automatisé permettrait une évolution positive
concernant plusieurs points :

\begin{description}
\item[Fiabilité/Constance de l’acquisition de données] : Actuellement les
niveaux sont relevés par le propriétaire des lieux. Une rationalisation
des moyens d’acquisition rendrait plus homogènes entre elles les
différentes mesures et ne dépendraient pas d’un facteur humain
subjectif.
\item[Rapidité de la transmission de données] : Le temps mis pour à la
fois mesurer les niveaux des réservoirs, transmettre en cas de
nécessité une demande à une société et chercher un agent de maintenance
proche du lieu à traiter peut grandement être amélioré par un système
automatique.
\item[Anticipation] : une gestion centralisée de l’acquisition de données
peut permettre une meilleurs anticipation des besoins des différents
sites (prévoir à l’avance du moment où il faudra accéder à un site)
\item[Optimisation des trajets] : La meilleur anticipation des besoins
permet de limiter la quantité de trajets à effectuer (et donc le coût
des maintenances). En effet, si on est capable d’anticiper les besoins
de maintenance, on peut envoyer un agent de maintenance sur plusieurs
sites en un trajet.
\end{description}
Cependant, tout cela se fait au prix d’un investissement de départ
important, car une installation doit être faite sur tous les sites. De
plus les sociétés de maintenance doivent toutes être équipées avec le
système automatisé.

Les enjeux principaux sont donc de limiter les coûts de la nouvelles
solution tout en permettant d’améliorer la qualité, la rapidité et de
baisser les coûts à long terme de la maintenance de ces sites.




\section{Annexes}
\subsection{Annexe 1 : Résultats étude de faisabilité}
Il s’agit dans cette annexe de présenter les différents résultats
détaillés des études, en proposant les références des diverses
recherches effectuées.

\subsubsection{Système embarqués}
http://arduino.cc/

Quelques caractéristiques globales des circuits imprimés Arduino :

\begin{description}
\item[Alimentation :] 12V
\item[Connectivités compatibles :] ethernet, série, parallèle, wifi, gsm,
usb, bluetooth
\item[Dimension maximales :] 10cm x 6cm
\item[Mémoire :] extensible
\item[Environnement :] multi-plateforme
\item[Particularité :] open source
\item[Prix :] 150€
\end{description}



L’étendue de l’offre Arduino (une quinzaine de cartes différentes)
permet de répondre au plus juste aux demandes liées aux particularités
du site.

\subsubsection{Production et gestion de l’énergie}
Sur chaque site, on peut distinguer deux types d’appareils électroniques
nécessitant une alimentation en énergie :
\begin{itemize}
\item les capteurs (plusieurs appareils en nombre limité fonctionnant
généralement sur 4V à 12V)
\item le système embarqué (appareil unique fonctionnant généralement en
12V ou 24V)
\end{itemize}

En étudiant les diverses solutions actuellement disponibles (2011), on
distingue trois types d’accès à l’énergie :

\paragraph{par stockage : accumulateurs (pile, batterie, ...),
condensateurs...}
L’installation est aisée et peu coûteuse mais cette gestion de l’énergie
nécessite une opération de maintenance régulière
(changement/chargement) pouvant s’avérer coûteuse à la longue.

\paragraph{par production autonome : énergies renouvelables (éolienne,
photovoltaïque, turbine, …) nucléaire, énergies fossiles (gaz, pétrole,
…)}
Une telle gestion de l’énergie assure l’autonomie complète du site mais
engendre généralement des coûts et/ou des travaux trop conséquents pour
ce genre de projet.

\paragraph{par convoyage : ERDF, ORES, ...}

Le raccordement à un réseau existant à proximité du site constitue un
accès standard à l’énergie mais se trouve mis à mal par le caractère de
réclusion des sites engendrant ainsi des coûts élevés pour assurer la
liaison.

\subsubsection{Capteurs}
Les capteurs que nous allons choisir doivent pouvoir résister à des
conditions extrêmes comme c’est le cas au Nord de la Norvège.

Nous devons donc sélectionner nos capteurs suivant leur résistance aux
hautes températures, à de fortes pression. Ils doivent également être
résistant au substances mesurées (rouille ou corrosion pour de l’eau ou
de l’essence).

Pour notre système, il nous faudra trouver des capteurs de niveau pour
mesurer les flux, des capteurs de températures et pression pour assurer
la sécurité de certaines cuves (à essence notamment), d’incendies, de
présence d’intrus.

\paragraph{Capteurs de niveau :}

Il y a différents types de capteurs de niveau plus ou moins adaptés au
produit mesuré :

\begin{itemize}
\item Radar
\item Radar à impulsion guidées
\item Ultrason
\item Capacitif
\item Hydrostatique
\item Radiométrie
\end{itemize}

Enterprises fournissant ce genre de capteurs :
\textbf{Vega}

Lien :
http://www.vega.com/fr/Mesure\_de\_niveau.htm

Il y a beaucoup de capteurs de niveau ayant une grande résistance au
températures extrêmes ( de - 200 à + 450 °C pour certains) et à de
hautes pressions et avec peut d'entretien à faire. La
précision de la mesure est de l’ordre de quelques millimètres.

Les capteurs de cette entreprise sont adaptés à une utilisation
industrielle, mais dans les exemples d’utilisation on peut les prendre
pour la prévention des crues, alors ça doit être utilisable pour le
projet.

Il y a aussi des capteurs spécialisés pour la mesure d’eau, de farine...

Adapté pour des mesures de niveau liquide et solide.

\textbf{SensorTechnics}
Lien : 
http://sensortechnics.domainfactory-kunde.de/fr/produits/capteurs-et-pressostats-de-niveaux/capteurs-hydrostatiques-pour-la-mesure-de-niveaux-de-liquides/

Capteurs globalement identiques à ceux de Vega.

Globalement une alimentation pouvant aller de 10 et 36 V par capteurs
avec des signaux de 0 à 20 mA en sortie à traiter (valeurs qui semblent
être standard

\textbf{Conclusion :}

Sur marché, on trouve de nombreux capteurs de mesure de niveau
s’adaptant à des mesures de produits liquides ou solides.

De nombreuses sociétés offres des produits résistants aux conditions
extrêmes en termes de températures et pression tout en offrant une
bonne précision de l’ordre de quelques millimètres.

En moyenne ces capteurs ont des besoins énergétiques de l’ordre de 10 à
36 V pour des signaux de sorties de 0 à 20 mA.

\paragraph{Détecteur de présence:}

Plusieurs types de capteurs ou de moyens de détection sont envisageables
:

\textbf{Capteurs de proximité capacitifs}

\begin{itemize}
\item détection selon la structures de matériaux (surtout pour métaux)
\item détection d’une distance 3 cm
\end{itemize}

Non adapté à notre cas

\textbf{Capteurs de proximité inductifs}

Détecte les matériaux conducteurs

Non adapté à notre cas

\textbf{Capteurs infra-rouges}

\begin{itemize}
\item détecte par infrarouge
\item inconvénients : dépend de la luminosité, températures de la cible
\end{itemize}
Peut être adapté à notre cas

\textbf{Capteurs de détection par pression}

\begin{itemize}
\item détecte quelque chose à l’endroit où est placé le capteur
\item inconvénients : en placer plusieurs, ne pas être trop sensible
\end{itemize}
Peut être adapté à notre cas

\textbf{Analyse d’image par webcam}
\begin{itemize}
\item demande une bonne luminosité
\item beaucoup de données à traiter
\item adapté pour les zones à haut risques de sécurité
\end{itemize}
Peut être adapté à notre cas

Nous choisirons des capteurs infra-rouges pour détecter l’arrivée
d’intrus notamment pour les sites stockant de l’essence.

On pourra combiner ceux-ci avec des caméras genre webcam pour en
améliorer la sécurité pour les lieux critiques.

Coût moyens : entre 15 et 70 € par capteurs

\paragraph{Capteurs de températures}

\textbf{Thermomètres infrarouges}
\begin{itemize}
\item coûts élevés
\end{itemize}
\textbf{Thermomètres à sonde}

http://www.proges.com/plug-and-track/surveillance-de-temperature-en-temp-reel/sondes-de-temperature-ethernet.html?gclid=CI-Dnqrcwq0CFaEhtAodADXrBA

\begin{itemize}
\item adaptés à plusieurs cas d’utilisation
\item large gamme de mesures (-100 à + 200 °C)
\item modèles compatibles sur réseau
\end{itemize}

Nous utiliserons des thermomètres à sondes bien adaptés à notre cas.

\paragraph{Capteurs de mesure de pression}

Il y a différents modèles de baromètre disponible sur le commerce.

On peut en distinguer 2 types différents : numérique ou analogique.

Nous choisirons le numériques qui est plus résistants aux conditions
extrêmes de températures et sera plus adapté pour le traitement des
données comparé à un analogique.

\subsubsection{Systèmes de communication}
Le système de communication regroupe :

\begin{itemize}
\item Les échanges entre les capteurs et le système embarqué (relevé de
niveaux)
\item La localisation des systèmes embarqués
\item L’envoi d’informations sur les relevés de niveaux du système
embarqué au serveur distant.
\item La récupération des demandes d’intervention
\item La localisation des agents de maintenance.
\end{itemize}

\paragraph{Relevé de niveaux}
Le relevé des niveaux des réservoirs s’effectue à l’aide de capteurs (cf
Partie Capteurs). Cette communication entre les capteurs et la station
doit respecter plusieurs critères. Elle doit : 

\begin{itemize}
\item être adaptée à toute condition météorologique
\item permettre de gérer les différents types de capteurs
\item permettre de réaliser une connexion entre deux points situés à une
centaine de mètre l’un de l’autre
\item être fiable
\item économique
\end{itemize}
Cette communication peut donc être réalisée à l’aide d’une connexion
filaire ou sans fil.

Chacun des deux modes de connexion présente ses avantages et
inconvénients. 

Par exemple, la connexion sans fil permet de s’affranchir des
contraintes de câblage et d'entretien des câbles, et
permet au système embarqué d’être mobile. En revanche, il faut être
capable de couvrir toute la zone, dans le cas où une solution axée sur
la technologie Wi-fi est retenue. En effet les capteurs peuvent être
distant de plusieurs dizaines de mètres les uns des autres.

L’utilisation d’une connexion de type filaire permet en revanche
d’assurer une meilleure fiabilité et un meilleur débit pour les
transmissions d’information. De plus, c’est peut-être une solution
moins coûteuse en terme d’infrastructure et elle est tout à fait
envisageable. La distance entre un réservoir et le système embarqué
n’est que d’une centaine de mètres au maximum. Mais elle ne respecte
pas la contrainte d’entretien lié aux différents types de météo et
climats possibles.

\paragraph{Localisation des systèmes embarqués}
Il sera utile d’utiliser la technologie GPS afin de localiser le système
embarqué (et donc le site), pour par exemple faciliter l’accès lors
d’une intervention. 

\subsubsection{Transmission des informations du système embarqué au serveur
distant}
Le système embarqué est chargé de faire parvenir les informations des
différents niveaux de cuves et de réservoirs au serveur central. 

Ces informations peuvent être envoyées simplement lorsque le besoin de
remplir ou de vider une cuve se fait ressentir. Mais cela consistait à
un des points faibles mis en valeur par l’étude de l’existant.

Finalement il est préférable d’avoir un suivi en temps réel sur le
niveau de toutes les cuves de chaque lieu, auquel cas le système devra
à intervalles réguliers envoyer ses données. Toujours est-il qu’il doit
y avoir une liaison entre le système et le serveur distant. 

Cette liaison est également pilotée par plusieurs contraintes :

\begin{itemize}
\item La communication doit être possible depuis n’importe quel endroit
du globe.
\item Elle doit supporter différentes conditions météorologique
\item Elle doit être économique sur le plan de l’énergie
\end{itemize}
Cette liaison pourrait être de type filaire. Mais dans notre cas, de
nombreux sites se situent dans des régions pratiquement inaccessibles.
L’installation d’un réseau de communication filaire serait une
contrainte énorme et engendrerait des coût dramatiquement élevés. 

Une solution idéale et tout à fait envisageable est d’utiliser le réseau
GSM (GPRS), qui permet de s’affranchir de toutes les contraintes
précédentes.

Plusieurs systèmes de ce type existent déjà et pourraient convenir à
notre cas d’étude.

\paragraph{Récupération des demandes d’intervention}
L’exploitation des données relatives aux différents sites peut se faire
via l’utilisation d’un logiciel spécifique. 

Chaque intervenant a accès à une interface de gestion des sites dont il
est en charge. Pour chaque site, un tableau de bord permet de suivre
l’ensemble des relevés, mesures, …

Quand un besoin d’intervention est nécessaire pour un site, l’agence de
maintenance est prévenue par l’intermédiaire de son interface de
monitoring, mais peut être également informée par mail.

\paragraph{Localisation et suivi des camions d’intervention}
Chaque agent de maintenance intervenant va pouvoir être suivi à l’aide
d’un système de localisation (cf Partie Systèmes de localisation). Cela
va permettre de gérer les transports de manière plus efficace et
d’éviter certains coûts liés aux déplacements inutiles. 

La localisation peut se faire facilement en équipant le poids lourds
d’un outil utilisant la technologie GPS par exemple, la difficulté
reposant sur le système de gestion global. La position GPS reçue peut
ainsi être communiquée au serveur central via le réseau GSM. 

Pour cela, on remarque qu’un certains nombre de solutions sont déjà
mises en oeuvre par différentes entreprises. Par exemple, la société
ELOSystèmes conçoit des systèmes embarqués de gestion de poids lourds.
Ce sont des systèmes simples à mettre en oeuvre.

\subsubsection{Systèmes de localisation}
Il y a deux éléments à géolocaliser : les agents de maintenance et les
sites à rejoindre. Les contraintes sont :

\begin{itemize}
\item Localisation à l’échelle planétaire
\item Précision permettant aux agents de maintenance de rejoindre les
sites (donc au pire des cas 500m)
\end{itemize}
\paragraph{Systèmes existants}
\begin{description}
\item[Wifi et RFID] : La géolocalisation doit se faire à l’échelle
planétaire, en zone peu accessible. Ceci élimine les localisations de
type Wifi ou RFID qui n’offrent qu’une faible portée.
\item[GPRS] : La localisation par le GPRS serait envisageable, cependant
la précision offerte par ce système (10km dans le pire des cas) n’est
pas suffisante.
\item[GPS] : Le GPS offre une portée à l’échelle planétaire et une
précision de 15 à 100m qui permet aux intervenants de rejoindre le site
en cas de besoin.
\end{description}
\paragraph{Pour les agents de
maintenance}
Le GPS est évidemment approprié pour les agents de maintenances. Ce
système est éprouvé pour les véhicules et bon marché. Sa consommation
est raisonnable pour son installation dans un véhicule d’un agent de
maintance.

\paragraph{Pour les sites}
Une localisation par GPS implique une consommation d’énergie non
négligeable. Hors dans la plupart des cas il n’est pas nécessaire
d’avoir une localisation dynamique, les sites sont immobiles. Il est
donc préférable de n’effectuer une géolocalisation des sites à
l’installation seulement et de conserver cette position. Cette première
géolocalisation sera effectuée par GPS à la mise en place du site.

Il faut aussi envisager la possibilité d’un site mobile, qui aurait
alors besoin d’un apport énergétique particulier.


\subsection{Annexe 2 : Bibliographie}
\subsubsection{Systèmes embarqués}
\begin{description}
\item[Arduino :] http://www.arduino.cc/
\end{description}
\subsubsection{RTOS}
\begin{description}
\item[LynxOS :] http://www.lynuxworks.com/
\item[RTLinux :] http://www.rtlinuxfree.com/
\item[Enea OSE :] http://www.enea.com/software/products/rtos/ose/
\item[QNX :] http://www.qnx.com/
\item[VxWorks :] http://www.windriver.com/products/vxworks/
\end{description}

\subsubsection{Systèmes de communication}
\begin{itemize}
\item Présentation système de logistique pour Camion (ELOsystème)
\end{itemize}

\end{document}

