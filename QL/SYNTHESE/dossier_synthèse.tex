\documentclass{mise_en_page}

\projet{Projet Ingéniérie}
\equipe{H4314}
\responsable{Tristan Delizy}
\redacteurs{Arnaud Lahache} % Optionnel
\titre{Dossier de Synthèse du projet COPEVUE-MONIT}
\version{1.0}
\objet{Ce dossier présentera la solution générale proposée par notre équipe dans le cadre du projet COPEVUE-MONIT}
\etat{draft} %draft, non relu, non validé, validé

\begin{document}

\maketitle

\begin{historique}
    \histo{0.1}{30/01/2012}{Initialisation du document}
    \histo{0.5}{06/01/2012}{Élaboration du contenu}
\end{historique}

\newpage

\tableofcontents

\section{Introduction}

\subsection{Objectif de ce dossier}
Le dossier de synthèse du projet COPEVUE-MONIT a pour objectif de présenter les principaux choix et les différentes élaborations de la solution proposée. Ce dossier aura pour but de présenter les nombreuses améliorations que notre solution pourrait impacter sur le système de la COPEVUE.

\section{Documents applicables et de référence}

\subsection{Documents de référence}

\begin{description}
	\item[QL / PAQP / PAQP] Plan d'assurance qualité du projet COPEVUE-MONIT, présentant notre démarche qualité au niveau \emph{projet}.
	\item[GLOSSAIRE / QA / glossaireQualite ] Glossaire spécifique à la démarche qualité du projet.
	\item[GLOSSAIRE / GL / glossaire ] Glossaire commun à l'ensemble du projet COPEVUE-MONIT.
\end{description}

\subsection{Documents applicables}

\begin{description}
	\item[AE / Etude de faisabilité] Dossier n°1 du projet COPEVUE-MONIT Présentant l'étude de faisabilité du projet.
	\item[ABTNS / STB ] Dossier n°2 du projet COPEVUE MONIT présentant la spécification technique des besoins du nouveau système.
	\item[DNS / Dossier de conception ] Dossier n°3 du projet COPEVUE-MONIT présentant la conception générale dans le cadre de la définition du nouveau système proposé.
\end{description}

\section{Terminologie et glossaire}

Les terminologies utilisées dans ce dossier sont spécifiques au projet COPEVUE-MONIT. Afin de pouvoir séparer les terminologies liées au projet général, ainsi que les terminologies seulement en relation avec la démarche qualité du projet, nous avons constitué deux documents constituant les glossaires vers lesquels vous pourrez vous référer :

\begin{description}
	\item[GLOSSAIRE / QL / glossaireQualite ] Glossaire spécifique à la démarche qualité du projet.
	\item[GLOSSAIRE / GL / glossaire ] Glossaire commun à l'ensemble du projet COPEVUE-MONIT.
\end{description}

Vous pouvez donc vous référer à l'ensemble de ces documents pour tous les points concernants les terminaisons ainsi que le glossaire.

\section{Contexte}

\section{Présentation de la solution proposée}

\subsection{Architecture}

\subsubsection{Architecture applicative}

Schéma de l'architecture applicative générale

\subsubsection{Architecture technique}

Schéma de l'architecture technique générale

\subsection{Fonctionement général}

\subsubsection{Capteurs}

\subsubsection{Géolocalisation}

\subsubsection{Système d'exploitation}

\subsubsection{Apports énergétique}

\subsubsection{Communication}

\subsection{Apports de la solution proposée}

\subsubsection{Thèmes de progrès}

\subsubsection{Coûts de la solution}

\section{Mise en place du système}

\subsubsection{Mise en place de l'architecture techique et applicative}

\subsubsection{Intégration avec le système existant}

\section{Conclusion}

\section{Annexes}

\end{document}
