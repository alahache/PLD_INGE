
\documentclass{mise_en_page}
\usepackage[utf8]{inputenc}
\usepackage[T1]{fontenc}
\usepackage[french]{babel}
\usepackage{amsmath}
\usepackage{amssymb,amsfonts,textcomp}
\usepackage{array}
\usepackage{supertabular}
\usepackage{hhline}

\projet{Projet Ingéniérie}
\equipe{H4314}
\responsable{Tristan Delizy}
\redacteurs{Arnaud Lahache} % Optionnel
\titre{Gestion de la documentation}
\version{1.3}
\objet{Ce dossier a pour but d’organiser la gestion ainsi que la production de
la documentation au sein du projet.}
\etat{validé} %draft, non relu, non validé, validé

\makeatletter
\newcommand\arraybslash{\let\\\@arraycr}
\makeatother
\setlength\tabcolsep{1mm}
\renewcommand\arraystretch{1.3}

\begin{document}

\maketitle

\begin{historique}
    \histo{1.0}{16/01/2012}{Draft initial}
    \histo{1.1}{23/01/2012}{Évolution du document}
    \histo{1.2}{30/01/2012}{Finalisation}
    \histo{1.3}{30/01/2012}{Validation}
\end{historique}

\newpage

\tableofcontents

\section{Introduction}

\emph{documentation :} ensemble de documents relatifs à un projet - notice -
mode d’emploi - action de sélectionner, classer, utiliser ou diffuser
des documents. (Source : Le Petit Larousse - 1994)

\section{Acteurs de la
gestion de documentation}

Les différents acteurs sont :

\begin{itemize}
\item Le responsable qualité (chargé de la gestion de la documentation,
de la vérification)
\item Le(s) auteur(s) du document
\item Le chef de projet
\end{itemize}

\begin{flushleft}
\tablehead{}
\begin{supertabular}{|m{5.347cm}|m{10.621cm}|}
\hline
\centering \textbf{ACTEUR} &
\centering\arraybslash \textbf{RESPONSABILITES}\\\hline
Responsable qualité &
\begin{itemize}
\item diffusion homogène des outils de production de documents\item
supervision du fonctionnement courant\item définition et respect des
règles d’identification, de structuration et de classement\item
contrôle de la cohérence et de l’homogénéité dans la gestion de la
documentation du projet\item relecture pour commentaires et
enrichissements éventuels\item contrôle de la conformité par rapport
aux standards méthodologiques (contenu attendu du document) et aux
règles de présentation et structuration des documents\end{itemize}
\\\hline
Auteur

(un document peut être produit par plusieurs auteurs) &
\begin{itemize}
\item objectifs et contenu du document\item gestion des mises à jour et
des versions-révisions successives du document\item sauvegarde du
document\end{itemize}
\\\hline
Chef de Projet

~
 &
\begin{itemize}
\item relecture pour commentaires et enrichissements éventuels\item
diffusion vers les destinataires identifiés du document\end{itemize}
\\\hline
\end{supertabular}
\end{flushleft}



\section{Cycle de vie d’un document}



Un document passe par un certain nombre d{\textquotesingle}états :

\begin{flushleft}
\tablehead{}
\begin{supertabular}{|m{4.236cm}|m{11.733cm}|}
\hline
\textbf{Etat} &
\textbf{Signification}\\\hline
travail &
le document est en cours d{\textquotesingle}élaboration par
l{\textquotesingle}auteur\\\hline
terminé &
le document satisfait l{\textquotesingle}auteur; il est prêt à être
diffusé\\\hline
vérifié (optionnel) &
le document est approuvé par d{\textquotesingle}autres membres de
l{\textquotesingle}équipe, des intervenants extérieurs et/ou le
contrôle qualité\\\hline
validé &
le document est approuvé par les personnes habilitées et prend valeur de
référence au sein du projet\\\hline
périmé &
le document n’est plus adapté et est donc retiré à tous ses détenteurs
(retrait d{\textquotesingle}usage)\\\hline
archivage &
le document n{\textquotesingle}est plus consulté régulièrement, mais une
trace de son existence demeure (pour une durée définie par le chargé de
gestion de la documentation du projet)\\\hline
destruction &
{}- le document n{\textquotesingle}est pas archivé ou

{}- le délai d{\textquotesingle}archivage est écoulé\\\hline
\end{supertabular}
\end{flushleft}



\section{Identification et structure des documents}
\subsection{Identification}
Cette partie concerne l’identification des livrables.




Chaque livrable est identifié par une seule référence dans le projet,
constituée de trois éléments:




\textbf{nom\_du\_projet / nature\_de\_document / identification\_du\_document}




\textbf{nom\_du\_projet}

\begin{itemize}
\item Le nom du projet est COPEVUE-MONIT
\end{itemize}



\textbf{nature\_de\_document}

\begin{itemize}
\item Identifié selon le type du document rédigé. Il existe différents
types pré-établis, mais vous pouvez si besoin en créer de nouveaux
s’ils s’avèrent inexistants :
\end{itemize}
\begin{enumerate}
\item \textbf{GESTION :} Documents relatifs à la Gestion de projet à destination
du CdP

→ Dossier d’init ; Dossier de synthèse
\item \textbf{QUALITE :} Documments relatifs à l’assurance Qualité du projet

→ Gestion de doc ; PAQP ; BP1\&2

\item \textbf{GLOSSAIRE :} Documents relatifs au glossaire

→ Glossaire projet

→ Glossaire qualité

\item \textbf{AE :} Analyse Exploratoire

→ Dossier d’étude de faisabilité

\item \textbf{ABTNS :} Analyse des Besoins Techniques du Nouveau Système

→ STB

\item \textbf{DNS :} Définition du Nouveau Système / Conception

→ Dossier de conception de nouveau système

\item \textbf{DSE :} Décomposition en Sous-Ensembles
\item \textbf{SI :} Spécification des Interfaces

→ Dossier d’interface de communication
\item \textbf{CdC :}  Cahier des Charges

→ CdC

\end{enumerate}

\textbf{identification\_du\_document}

\begin{itemize}
\item Identification claire du nom du projet. (Ex : Dossier
d’Initalisation, PAQL)
\end{itemize}



Les deux premiers éléments (nom\_du\_projet et nature\_de\_document)
constitueront des dossiers physiques, alors que le troisième élément
(identification\_du\_document) correspondra au nom physique du fichier.




\subsection{Structure}
Il est convenu que tout document doit comporter les éléments suivants
sur la page de garde :

\begin{itemize}
\item le titre du document,
\item la référence du document,
\item la date de dernière mise à jour,
\item le nom de l’auteur (ou des auteurs),
\item l’objet du document (présentation rapide du contenu)
\end{itemize}



D’autre part, sur chaque page du document préciser :

\begin{itemize}
\item le titre du document,
\item la référence,
\item le numéro de page / nombre de pages total.
\end{itemize}



Pour chaque document, une Bibliographie doit figurer afin de répertorier
toutes les sources et la gestion des versions

\section{Gestion des révisions}



\begin{itemize}
\item La gestion des révisions des drafts se fera via la plateforme
Google Docs
\end{itemize}
\begin{itemize}
\item La gestion des révisions des livrables se fera à l’aide du
logiciel Git et de la plateforme github.
\end{itemize}



\subsection{Utilisation de
Google Docs}
Les auteurs des drafts devront posséder un compte sur Google afin de
profiter des services offerts par l’outil Google Docs.




Google Docs propose directement une gestion automatique des révisions
pour chaque document créé. Il propose également un outil de discussion
instantanée intégré, ce qui permet une collaboration encore plus
interactive.

\subsection{Utilisation de github}
Les auteurs d’un document devront créer un compte sur le plateforme
github : github.com 




Ensuite, il faudra que les auteurs rejoignent le dépôt suivant :

https://github.com/alahache/PLD\_INGE




\subsection{Utilisation de git}




L’utilisation de Git permettra d’effectuer un versionnage efficace des
livrables. Pour installer Git sur ubuntu, il suffit de taper la
commande :

sudo apt-get install git




Afin d’en apprendre plus sur l’utilisation de Git, vous pourrez vous
référer à l’Annexe n°B : Tutoriel d’utilisation de Git

\section{Production des documents}
\subsection{Production des drafts}
Les Drafts seront produits au moyen du service proposé par Google Docs.
En effet, cet outil propose une collaboration instantanée entre les
différents auteurs du document, ce qui permet d’accélérer
significativement la production.

\subsection{Production des livrables}
\paragraph{Méthode de production planifiée}
Les Livrables seront produits grâce à l’outil Latex, permettant une mise
en forme automatique basée sur des règles de génération de documents
pré-établies à l’avance.




Un fichier de mise en forme préalablement réalisé par le Responsable
Qualité sera fourni avant la réalisation de chaque document.




Avantages de la mise en place de Latex :

\begin{itemize}
\item La rédaction est portée sur le contenu et non la mise en page du
document
\item Cette simplicité augmente la productivité
\item Le format texte des document .tex permet un versionnage beaucoup
plus efficace
\item Le mode de production de Latex permet de générer automatiquement
l’ensemble des documents en fichiers PDF prêts à être imprimés
\end{itemize}



Installer latex sur windows :

http://miktex.org/

\paragraph{Risques engendrés}
L’utilisation de Latex, bien qu’apportant de nombreux avantages,
présente également quelques risques :

\begin{itemize}
\item L’utilisation du langage Latex peut dérouter, voire rebuter
certains rédacteurs
\item Il convient d’utiliser un système adapté (linux, voire des
logiciels spécifiques) afin de pouvoir rédiger en Latex
\end{itemize}
\paragraph{Méthode de
production <<de secours>>}
Dans le cas où la méthode planifiée avec Latex ne porterai pas ses
fruits, une génération des documents à l’aide d’un éditeur de texte
classique (du type LibreOffice) sera préférée. Le choix final de la
méthode de production sera effectué le plus tôt possible en fonction
des résultats produits avec Latex.




Du fait du risque potentiel engendré par Latex, les résultats du choix
de la méthode de production seront rapportés dans le dossier de Bilan
FORMELLE, rédigé par le Responsable Qualité.

\section{Annexes}
\subsection{Annexe A : Modèle de rédaction d’un livrable en Latex}
\emph{Vous devez posséder le fichier de mise en page “mise\_en\_page.cls” dans le
même dosser que votre fichier .tex}

Un modèle basique de document tex utilisant ce modèle se résume à :

\begin{flushleft}
\tablehead{}
\begin{supertabular}{|m{16.345cm}|}
\hline
{\textbackslash}documentclass\{mise\_en\_page\}

~

{\textbackslash}responsable\{...\}

{\textbackslash}redacteurs\{...\} \% Optionnel

{\textbackslash}titre\{Titre du document\}

{\textbackslash}version\{0.1\}

{\textbackslash}objet\{Description rapide du document\}

{\textbackslash}etat\{draft\} \%draft, non relu, non validé, validé

~

{\textbackslash}begin\{document\}

~

{\textbackslash}maketitle

~

{\textbackslash}begin\{historique\}

\ \ {\textbackslash}histo\{0.1\}\{xx/xx/2012\}\{Draft initial\}

{\textbackslash}end\{historique\}

~

{\textbackslash}tableofcontents

{\textbackslash}newpage

~

{\textbackslash}section\{Première partie\}

~

{\textbackslash}end\{document\}\\\hline
\end{supertabular}
\end{flushleft}



\subsection{Annexe B : Tutoriel d’utilisation de Git}



\subsection{Installation de Git}



{}- Téléchargez et installez git :

\ \ {\textgreater} sudo apt-get install git-core







{}- Inscrivez-vous sur github :

\ \ https://github.com/







{}- Configurez git :

\ \ {\textgreater} git config -{}-global user.name
{\textquotedbl}alahache{\textquotedbl}

\ \ {\textgreater} git config -{}-global user.email
arnaud.lahache@gmail.com







{}- Récupérez votre clé publique ssh :

\ \ [http://help.github.com/linux-key-setup/]




\ \ \ \ Pour pouvoir vous connecter à github, le site utilise ssh et a
besoin

\ \ de votre clé publique RSA. Il suffit d{\textquotesingle}ajouter la
clé publique sur la page

\ \ {\textquotedbl}account{\textquotedbl} (https://github.com/account),
dans la section {\textquotedbl}SSH Public Keys{\textquotedbl} :

\ \ cliquez sur {\textquotedbl}add another public key{\textquotedbl} et
copiez/collez le contenu de votre

\ \ clé publique. Pour vérifier si vous avez déjà une clé publique,
allez dans le

\ \ répertoire \~{}/.ssh :




\ \ {\textgreater} cd \~{}/.ssh

\ \ {\textgreater} ls




\ \ {}- Si le répertoire .ssh et les fichiers id\_rsa et id\_rsa.pub
existent, vous

\ \ avez donc déjà une clé publique. Copiez-collez donc le contenu de
id\_rsa.pub dans

\ \ la page citée plus haut




\ \ {}- Si le répertoire .ssh ou les fichiers
n{\textquotesingle}existent pas, il faut génerer la clé :




\ \ {\textgreater} ssh-keygen -t rsa -C
{\textquotedbl}arnaud.lahache@gmail.com{\textquotedbl}




\ \ Le programme voudra vous demander un fichier ou une passphrase, mais
appuyez

\ \ simplement sur {\textquotesingle}entrée{\textquotesingle} pour
ignorer... Une fois la clé générée, copiez-collez le

\ \ contenu de id\_rsa.pub sur la page citée plus haut.




\ \ {}- Pour tester que votre clé a bien été prise en compte, lancez la
commande :




\ \ {\textgreater} ssh git@github.com




\ \ Si la connection vous retourne un message du type :




\begin{flushleft}
\tablehead{}
\begin{supertabular}{|m{16.345cm}|}
\hline
The authenticity of host {\textquotesingle}github.com
(207.97.227.239){\textquotesingle} can{\textquotesingle}t be
established.

RSA key fingerprint is 16:27:ac:a5:76:28:2d:36:63:1b:56:4d:eb:df:a6:48.

Are you sure you want to continue connecting (yes/no)? yes\\\hline
\end{supertabular}
\end{flushleft}



\ \ Mettez {\textquotedbl}yes{\textquotedbl} puis appuyez sur
{\textquotesingle}entrée{\textquotesingle}




\ \ Si vous voyez un message du type :




\begin{flushleft}
\tablehead{}
\begin{supertabular}{|m{16.345cm}|}
\hline
Hi alahache! You{\textquotesingle}ve successfully authenticated, but
GitHub does not provide shell access.

Connection to github.com closed.\\\hline
\end{supertabular}
\end{flushleft}






\ \ Alors votre clé a bien été prise en compte.







Pour créer un dépôt sur git : (pas très utile pour le projet)

[http://help.github.com/creating-a-repo]




Pour forker le projet :

[http://help.github.com/forking/]




\subsubsection{Rejoindre le Dépôt}



{}- Allez sur la page :

\ \ https://github.com/alahache/PLD\_INGE

\ \ \ \ et cliquez sur {\textquotedbl}Watch{\textquotedbl}







{}- Placez-vous dans un dossier où vous souhaitez ajouter le dépôt :

\ \ {\textgreater} mkdir PLD\_INGE

\ \ {\textgreater} cd PLD\_INGE







{}- Clonez le dépôt depuis l{\textquotesingle}adresse du repository :

\ \ {\textgreater} git clone git@github.com:alahache/PLD\_INGE.git




\ \ Cette commande permet de récupérer le dépot se situant à
l{\textquotesingle}adresse

\ \ git@github.com:alahache/PLD\_INGE.git










3 - Utiliser git

{}-{}-{}-{}-{}-{}-{}-{}-{}-{}-{}-{}-{}-{}-{}-{}-




{}- Ajouter un nouveau fichier :

\ \ {\textgreater} touch nouveau\_fichier

\ \ {\textgreater} git add nouveau\_fichier







{}- Ajouter tous les nouveaux fichiers :

\ \ {\textgreater} git add .







{}- Effectuer un commit local :

\ \ {\textgreater} git commit -m {\textquotedbl}ajout du fichier
nouveau\_fichier{\textquotedbl}







{}- Voir quel est l{\textquotesingle}état en cours de votre dépôt local
:

\ \ (ex : fichiers pas encore ajoutés, fichiers édités, conflits, etc.)

\ \ {\textgreater} git status







{}- Uploader le dépot local sur github :

\ \ [http://help.github.com/remotes/]

\ \ Si c{\textquotesingle}est la première fois que vous voulez uploader
(commande git push), faites :

\ \ {\textgreater} git remote add origin
git@github.com:alahache/PLD\_INGE.git




\ \ Cela permettera de pouvoir utiliser
{\textquotedbl}origin{\textquotedbl} à la place de
l{\textquotesingle}adresse complète

\ \ {\textquotedbl}git@github.com:alahache/PLD\_INGE.git{\textquotedbl}
à chaque fois que vous voulez faire

\ \ un push.




\ \ {\textgreater} git push origin master

\ \ Cette commande uploade la branche
{\textquotedbl}master{\textquotedbl} (principale) vers le remote
{\textquotedbl}origin{\textquotedbl}




\ \ ou tout simplement :

\ \ {\textgreater} git push







{}- Editer un fichier :

\ \ Effectuez des modifications sur un fichier déjà ajouté précédemment
(édition,

\ \ suppression)...

\ \ Puis utilisez la commande git commit avec l{\textquotesingle}option
-a pour commiter automatiquement

\ \ tous les fichiers modifiés :

\ \ {\textgreater} git commit -a -m {\textquotedbl}modification,
etc.{\textquotedbl}




\ \ Cela ne marche pas si vous venez de créer le fichier, il faut
d{\textquotesingle}abord utiliser git add.







{}- Mettre à jour le dépôt local :

\ \ {\textgreater} git pull origin master

\ \ Met à jour la branche master à partir du dépôt distant origin (cf.
git remote)







{}- Planifier la suppression d{\textquotesingle}un fichier :

\ \ {\textgreater} git rm fichier\_a\_supprimer




[http://help.github.com/git-cheat-sheets/]

\subsubsection{Utilisation des branch}



 {}- Créer une nouvelle branche (par exemple
{\textquotedbl}dev{\textquotedbl})

\ \ {\textgreater} git branch dev




 {}- Switcher vers la branche {\textquotedbl}dev{\textquotedbl}

\ \ {\textgreater} git checkout dev




 {}- Effectuez des modifications sur {\textquotedbl}dev{\textquotedbl},
effectuez un commit...

\ \ {\textgreater} git commit -a -m
{\textquotedbl}modifs...{\textquotedbl}




 {}- Envoyez les modifications vers GitHub via push (si la branche
{\textquotedbl}dev{\textquotedbl} n{\textquotesingle}est pas sur
GitHub) :

\ \ {\textgreater} git push origin dev

\ \ Ici push envoie les modifications vers le dépôt origin, vers la
branche {\textquotedbl}dev{\textquotedbl}. Si la

\ \ branche dev n{\textquotesingle}existe pas sur GitHub, elle sera
créée.




 {}- Envoyez les modifications vers GitHub via push (a la suite) :

\ \ {\textgreater} git push

\ \ Même si vous êtes dans master, il envoiera également les
modifications de {\textquotedbl}dev{\textquotedbl}




 {}- Revenir vers la branche master

\ \ {\textgreater} git checkout master

\ \ Vous remarquerez que vos modifications ne seront plus visibles




 {}- Fusionner la branche master avec les modiciations dans
{\textquotedbl}dev{\textquotedbl}

\ \ {\textgreater} git merge dev

\ \ {\textgreater} git push




 {}- Récupérer les modifications de la branche
{\textquotedbl}dev{\textquotedbl} dans la branche locale courante : (de

 préférence {\textquotedbl}dev{\textquotedbl})

\ \ {\textgreater} git pull origin dev




\subsubsection{Aller plus loin}



{}- Gérer différentes branches dans un même dépot (pour par exemple
faire des modifications expérimentales) :

\ \ http://www.alexgirard.com/git-book/3\_usage\_basique\_des\_branches\_et\_des\_merges.html




{}- Pour plus d{\textquotesingle}informations - documentation très
complète sur git :

\ \ http://www.alexgirard.com/git-book/index.html




\subsection{Annexe C : Plans Types}
\subsubsection{Dossier d’étude de faisabilité}
1. Introduction

2. Analyse de l’existant

\ \ 2.a. Analyse du métier

\ \ 2.b. Analyse des savoir-faire et des processus

\ \ 2.c. Analyse du matériel utilisé

3. Etude de faisabilité

\ \ 3.a. Synthèse sur système embarqué

\ \ 3.b. Synthèse sur gestion de l’énergie

\ \ 3.c. Synthèse sur capteurs

\ \ 3.d. Synthèse systèmes de communication

\ \ 3.e. Synthèse sur systèmes de localisation

\ \ …

4. Conclusions

Annexe 1 : Résultats détaillés de l’étude de faisabilité

\subsubsection{Dossier de spécification technique des besoins du système}
Objectif du document : Présenter les axes de progrès envisagés sur le
système existant, définir les exigences fonctionnelles et non
fonctionnelles, de manière à répondre aux besoins du clients et à
l’appel d’offre.

1. Introduction

2. Axes d’amélioration retenus

\ \ 2.1. Axes de progrès retenus

\ \ 2.2. Axes de progrès marginaux

\ \ 2.3. Faux axes de progrès éventuels

\ \ …

3. Description des exigences non fonctionnelles de votre futur système

4. Description des exigences fonctionnelles de votre futur système

(parmi toutes les exigences non fonctionnelles du §3, montrer, pour
chaque fonctionnalité, les exigences non fonctionnelles concernant
cette fonctionnalité, en étant PLUS PRECIS (raffinement)

5. Esquisse du système et impacts de la solution

\ \ 5.1. Esquisse du système avant de faire la conception

\ \ 5.2. Impacts de la solution

6. Bilan des améliorations

7. Conclusion

\subsubsection{Dossier de conception du nouveau système}
Objectif : Proposer une solution (conception système, architecture...)
qui réponde parfaitement aux exigences et à l’appel d’offre. Cette
solution doit prendre en compte les cas nominaux mais aussi les cas
non-nominaux (mode dégradé suite à des incidents).

1. Introduction

2. Organisation générale du système

3. Règles de pilotage du système

4. Architecture applicative

5. Architecture informatique \& matérielle (techique)

{}-{\textgreater} Ne pas oublier réseau de communication, les logiciels

6. Réflexions sur les données

(Modèles de données du système, modèles de la BDD intégrant les données
de la production, volumétrie (évaluation de la BDD pour une entreprise
type en s’appuyat sur réflexion de l’annexe 1, traffic sur le
réseau...)

7. Gestion des problèmes \& anomalies, sécurité

8. Conclusion

Annexe 1 : Représentation informatique des objets (messages...)

Annexe 2 : Réflexion sur le réseau

(principes qualité, pour la conception du réseau, description du réseau
(logiciel, couches)

Annexe 3 : Démarrage du système

\subsubsection{Dossier d’interface de communication}
Objectif : Ce dossier doit présenter les interfaces (API) entre les
différents “Sous-ensembles” du projet qui ont été définis lors de
l’approche produit. On proposera 3 modélisations possibles de cette
communication entre “Sous-ensemble” :

\begin{itemize}
\item Approche “urbanisation des SI” (blocs et sous-blocs)
\item Approche objet sous forme de “Diagramme de collaboration” (Modèle
UML)
\item Approche diagramme de flux (SA/RT - Approche fonctionnelle)
\end{itemize}
1. Découpage du projet

\ \ 1.1. Vue d’ensemble du projet (schéma montrant les relations)

\ \ 1.2. Division en sous-ensembles (ou blocs)

\ \ 1.3. Division en niveau inférieur (càd sous-sous-ensembles ou
sous-blocs)

2. Description de chacun des sous-ensembles

\ \ 2.1. Sous-ensemble Machine-Outils

\ \ \ \ 2.1.1. Sous-sous-ensemble X

\ \ \ \ 2.1.2. Sous-sous-ensemble Y...

\ \ 2.2. Sous-ensembles MOCN

\ \ 2.3. ….

3. Définition ces interfaces

\ \ 3.1. Relation entre les sous-ensembles (ou en dessous)

\ \ 3.2. Interfaces entre les sous-ensembles (ou en dessous)

4. Description détaillée

\ \ 4.1. Bloc Machine-Outils

\ \ \ \ 4.1.1. Sous-sous-ensembles X

\ \ \ \ 4.1.2. Sous-sous-ensembles Y

\ \ 4.2. Bloc BDD

\ \ 4.3. …




\subsubsection{Dossier de synthèse}
Objectif : Le dossier de synthèse a pour objectif de promouvoir le
résultat de l’analyse et de la conception du futur système. Ce dossier
s’adresse au comité de pilotage de la COPEVUE. Dans le cas de l’appel
d’offre lancé par ce dernier, ce dossier de synthèse a pour but d’aider
la décision du comité de pilotage quant à la décision de retenir la
réponse proposée par notre société.

1. Introduction

\ \ 1.1. Cadre de l’étude et objectifs

\ \ 1.2. Rappel synthétique des critiques de l’existant

2. Présentation de la solution proposée

\ \ 1.1. Architecture

\ \ 1.2. Fonctionnelement général

\ \ 1.3. Apports de la solution proposée

\ \ \ \ 1.3.1. Thèmes de progrès

\ \ \ \ 1.3.2. Coûts de la solution

3. Mise en place du système

\ \ 3.1. Description de la mise en place du système

\ \ 3.2. Plan de formation

\ \ 3.3. ROI

4. Annexes

\ \ 4.1. Organisation générale du système

\ \ 4.2. ...
\end{document}
