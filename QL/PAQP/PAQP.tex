\documentclass{mise_en_page}

\usepackage{booktabs} % Required for the \toprule, \midrule and \bottomrule lines
\usepackage{array}

\projet{Projet Ingéniérie}
\equipe{H4314}
\responsable{Tristan Delizy}
\redacteurs{Arnaud Lahache} % Optionnel
\titre{Plan d'Assurance Qualité Projet}
\version{1.3}
\objet{Ce dossier a pour but de décrire les dispositions prises en matière d’assurance de la qualité pour le projet COPEVUE-MONIT}
\etat{draft} %draft, non relu, non validé, validé

\begin{document}

\maketitle

\begin{historique}
    \histo{1.0}{23/01/2012}{Initialisation du document}
    \histo{1.1}{30/01/2012}{Début de la mise en page}
\end{historique}

\newpage

\tableofcontents

\section{Objet et caractéristiques du PAQP}

\subsection{Objectifs}
L’objectif du PAQP est de décrire les dispositions prises en matière d’assurance de la qualité pour le projet COPEVUE-MONIT. Ces dispositions seront à respecter pour l'ensemble des sous-projets. Cependant dans le cas où certains sous-projets auraient des contraintes spécifiques, ils pourront disposer de leurs propres démarches qualités spécifique dictées dans des PAQL au niveau \emph{Logiciel}.

Ce dossier aura donc pour but de décrire les grandes lignes d'assurance qulité afin d'assurer à la MOA une certaine qualité des livrables pour l'ensemble du projet.

\subsection{Documents applicables et de référence}

\subsubsection{Documents de référence}

\begin{description}
	\item[GLOSSAIRE / QA / glossaireQualite ] Glossaire spécifique à la démarche qualité du projet.
	\item[GLOSSAIRE / GL / glossaire ] Glossaire commun à l'ensemble du projet MIME.
\end{description}

\subsubsection{Documents applicables}

\begin{description}
	\item[QL / GDOC / gestionDocumentation] Gestion de la documentation générale de notre société. Ce dossier déjà réalisé servira de base à la gestion de documentation du PAQP.
\end{description}

\subsection{Terminologie et glossaire}

Les terminologies utilisées dans ce dossier sont spécifiques au projet COPEVUE-MONIT. Afin de pouvoir séparer les terminologies liées au projet général, ainsi que les terminologies seulement en relation avec la démarche qualité du projet, nous avons constitué deux documents constituant les glossaires vers lesquels vous pourrez vous référer :

\begin{description}
	\item[GLOSSAIRE / QL / glossaireQualite ] Glossaire spécifique à la démarche qualité du projet.
	\item[GLOSSAIRE / GL / glossaire ] Glossaire commun à l'ensemble du projet COPEVUE-MONIT.
\end{description}

Vous pouvez donc vous référer à l'ensemble de ces documents pour tous les points concernants les terminaisons ainsi que le glossaire.

\section{Objectifs et engagements qualité du projet}

\subsection{Exigences non fonctionelles}

L'appel d'offre émis par la COPEVUE nous a permis de dégager un ensemble d'exigences non fonctionnelles qui formeront les principaux objectifs 

\subsection{Déclinaison en engagements qualité}

\section{Conduite de projet}

\subsection{Présentation des relations et des rôles}

\subsubsection{Équipe SIMME}

Les responsabilités ont été distribuées au début du projet. Un chef de projet a été nommé ainsi qu'un responsable qualité. Le reste de l'équipe compose le GEI (Groupe d'Etude Informatique) (voir le tableau \ref{tab_equipe} plus bas.)

La responsabilité du chef de projet est de mener le projet à bien et de gérer son équipe, mais également de garder un \oe{}il sur la cohérence du contenu des différents dossiers. Le responsable qualité aura pour tâche d'assurer une certaine cohérence formelle des livrables rendus mais également de mettre en place une démarche qualité au niveau projet et sous-projet.

La responsabilité du GEI s'étend de l'étude jusqu'à la rédaction complète des drafts et des livrables des différents dossiers nécessaires au projet. C'est le GEI qui délivrera grâce à ses recherches la solution proposée.

\begin{table}[h]
	\centering
	\begin{tabular}{l r}
		\toprule
		\textbf{Prénom et Nom} & \textbf{Rôle dans l'équipe}\\
		\toprule
		Tristan Delezy & Chef de Projet \\
		Arnaud Lahache & Responsable qualité \\
		\midrule
		Jérémy Tuloup & GEI \\
		Pierre Lulé & GEI \\
		Léo Lefebvre & GEI \\
		Sébastien Laurent & GEI \\
		\bottomrule
	\end{tabular}
	\caption{\label{tab_equipe} Présentation de l'équipe}
\end{table}

\subsubsection{MOE}

Suivant les sous-projets développés lors du projet COPEVUE-MONIT, le développement pourra se faire en interne ou en externe. Dans tous les cas, l'équipe MOE constituée pour réaliser le sous-projet devra travailler en collaboration étroite avec la MOA (Équipe SIMME).



\section{Gestion de la documentation}



\section{Gestion de la configuration}

\section{Gestion des modifications}

\end{document}
