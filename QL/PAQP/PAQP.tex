\documentclass{mise_en_page}

\projet{Projet Ingéniérie}
\equipe{H4314}
\responsable{Tristan Delizy}
\redacteurs{Arnaud Lahache} % Optionnel
\titre{Plan d'Assurance Qualité Projet}
\version{1.3}
\objet{Ce dossier a pour but de décrire les dispositions prises en matière d’assurance de la qualité pour le projet MIME}
\etat{draft} %draft, non relu, non validé, validé

\begin{document}

\maketitle

\begin{historique}
    \histo{1.0}{23/01/2012}{Initialisation du document}
    \histo{1.1}{30/01/2012}{Début de la mise en page}
\end{historique}

\newpage

\tableofcontents

\section{Objet et caractéristiques du PAQP}

\subsection{Objectifs}
L’objectif du PAQP est de décrire les dispositions prises en matière d’assurance de la qualité pour le projet MIME. Ces dispositions seront à respecter pour l'ensemble des sous-projets. Cependant dans le cas où certains sous-projets auraient des contraintes spécifiques, ils pourront disposer de leurs propres démarches qualités spécifique dictées dans des PAQL au niveau \emph{Logiciel}.

Ce dossier aura donc pour but de décrire les grandes lignes d'assurance qulité afin d'assurer à la MOA une certaine qualité des livrables pour l'ensemble du projet.

\subsection{Documents applicables et de référence}

\subseubsection{Documents de référence}

\begin{description}
	\item[GLOSSAIRE / QA / glossaireQualite ] Glossaire spécifique à la démarche qualité du projet.
	\item[GLOSSAIRE / GL / glossaire ] Glossaire commun à l'ensemble du projet MIME.
\end{description}

\subseubsection{Documents applicables}

\begin{description}
	\item[QL / GDOC / gestionDocumentation] Gestion de la documentation générale de notre société. Ce dossier déjà réalisé servira de base à la gestion de documentation du PAQP.
\end{description}

\subsection{Terminologie et glossaire}

Les terminologies utilisées dans ce dossier sont spécifiques au projet MIME. Afin de pouvoir séparer les terminologies liées au projet général, ainsi que les terminologies seulement en relation avec la démarche qualité du projet, nous avons constitué deux documents constituant les glossaires vers lesquels vous pourrez vous référer :

\begin{description}
	\item[GLOSSAIRE / QL / glossaireQualite ] Glossaire spécifique à la démarche qualité du projet.
	\item[GLOSSAIRE / GL / glossaire ] Glossaire commun à l'ensemble du projet MIME.
\end{description}

Vous pouvez donc vous référer à l'ensemble de ces documents pour tous les points concernants les terminaisons ainsi que le glossaire.

\section{Objectifs et engagements qualité du projet}

\subsubsection{Exigences non fonctionelles}

L'appel d'offre émis par la COPEVUE nous a permis de dégager un ensemble d'exigences non fonctionnelles qui formeront les principaux objectifs 

\subsubsection{Déclinaison en engagements qualité}

\section{Conduite de projet}

\subsection{Présentation des relations et des rôles}


\section{Gestion de la documentation}



\section{Gestion de la configuration}

\section{Gestion des modifications}

\end{document}
