\documentclass{mise_en_page}

\projet{Projet Ingéniérie}
\equipe{H4314}
\responsable{Tristan Delizy}
\redacteurs{Arnaud Lahache} % Optionnel
\titre{Plan d'Assurance Qualité Projet}
\version{1.3}
\objet{Ce dossier a pour but de décrire les dispositions prises en matière d’assurance de la qualité pour le projet COPEVUE-MONIT}
\etat{draft} %draft, non relu, non validé, validé

\begin{document}

\maketitle

\begin{historique}
    \histo{1.0}{23/01/2012}{Initialisation du document}
    \histo{1.1}{30/01/2012}{Début de la mise en page}
\end{historique}

\newpage

\tableofcontents

\section{Objet et caractéristiques du PAQP}

\subsection{Objectifs}
L’objectif du PAQP est de décrire les dispositions prises en matière d’assurance de la qualité pour le projet COPEVUE-MONIT

\subsection{Documents de référence}

\subsection{Terminologie et glossaire}

\section{Objectifs et engagements qualité du projet}

\section{Conduite de projet}

\section{Gestion de la documentation}

\section{Gestion de la configuration}

\section{Gestion des modifications}

\end{document}
